% Options for packages loaded elsewhere
\PassOptionsToPackage{unicode}{hyperref}
\PassOptionsToPackage{hyphens}{url}
%
\documentclass[
]{article}
\usepackage{amsmath,amssymb}
\usepackage{iftex}
\ifPDFTeX
  \usepackage[T1]{fontenc}
  \usepackage[utf8]{inputenc}
  \usepackage{textcomp} % provide euro and other symbols
\else % if luatex or xetex
  \usepackage{unicode-math} % this also loads fontspec
  \defaultfontfeatures{Scale=MatchLowercase}
  \defaultfontfeatures[\rmfamily]{Ligatures=TeX,Scale=1}
\fi
\usepackage{lmodern}
\ifPDFTeX\else
  % xetex/luatex font selection
\fi
% Use upquote if available, for straight quotes in verbatim environments
\IfFileExists{upquote.sty}{\usepackage{upquote}}{}
\IfFileExists{microtype.sty}{% use microtype if available
  \usepackage[]{microtype}
  \UseMicrotypeSet[protrusion]{basicmath} % disable protrusion for tt fonts
}{}
\makeatletter
\@ifundefined{KOMAClassName}{% if non-KOMA class
  \IfFileExists{parskip.sty}{%
    \usepackage{parskip}
  }{% else
    \setlength{\parindent}{0pt}
    \setlength{\parskip}{6pt plus 2pt minus 1pt}}
}{% if KOMA class
  \KOMAoptions{parskip=half}}
\makeatother
\usepackage{xcolor}
\usepackage[margin=1in]{geometry}
\usepackage{longtable,booktabs,array}
\usepackage{calc} % for calculating minipage widths
% Correct order of tables after \paragraph or \subparagraph
\usepackage{etoolbox}
\makeatletter
\patchcmd\longtable{\par}{\if@noskipsec\mbox{}\fi\par}{}{}
\makeatother
% Allow footnotes in longtable head/foot
\IfFileExists{footnotehyper.sty}{\usepackage{footnotehyper}}{\usepackage{footnote}}
\makesavenoteenv{longtable}
\usepackage{graphicx}
\makeatletter
\newsavebox\pandoc@box
\newcommand*\pandocbounded[1]{% scales image to fit in text height/width
  \sbox\pandoc@box{#1}%
  \Gscale@div\@tempa{\textheight}{\dimexpr\ht\pandoc@box+\dp\pandoc@box\relax}%
  \Gscale@div\@tempb{\linewidth}{\wd\pandoc@box}%
  \ifdim\@tempb\p@<\@tempa\p@\let\@tempa\@tempb\fi% select the smaller of both
  \ifdim\@tempa\p@<\p@\scalebox{\@tempa}{\usebox\pandoc@box}%
  \else\usebox{\pandoc@box}%
  \fi%
}
% Set default figure placement to htbp
\def\fps@figure{htbp}
\makeatother
\setlength{\emergencystretch}{3em} % prevent overfull lines
\providecommand{\tightlist}{%
  \setlength{\itemsep}{0pt}\setlength{\parskip}{0pt}}
\setcounter{secnumdepth}{-\maxdimen} % remove section numbering
\usepackage{bookmark}
\IfFileExists{xurl.sty}{\usepackage{xurl}}{} % add URL line breaks if available
\urlstyle{same}
\hypersetup{
  pdftitle={BIOSTAT 625 Report},
  pdfauthor={Naman Dhariwal},
  hidelinks,
  pdfcreator={LaTeX via pandoc}}

\title{BIOSTAT 625 Report}
\author{Naman Dhariwal}
\date{2025-12-11}

\begin{document}
\maketitle

\begin{enumerate}
\def\labelenumi{\arabic{enumi}.}
\tightlist
\item
  Introduction
\end{enumerate}

This project predicts 5-year all-cause mortality among U.S. smokers
using a Light Gradient Boosting Machine (LightGBM) classifier applied to
the NLMS Tobacco Use Cohort. The dataset contains nearly 500,000
individuals with rich demographic, socioeconomic, and behavioral
predictors. Because mortality is a rare event (≈4\%), the modeling
approach must address severe class imbalance while preserving
sensitivity to at-risk individuals. LightGBM was selected for its strong
performance on large, imbalanced tabular data and its compatibility with
model interpretability tools such as SHAP.

\begin{enumerate}
\def\labelenumi{\arabic{enumi}.}
\setcounter{enumi}{1}
\tightlist
\item
  Data \& Preprocessing
\end{enumerate}

The outcome variable is a binary indicator of death within 5 years
(inddea). Predictors include:

\begin{itemize}
\tightlist
\item
  Demographics: age (age), sex (sex), race/ethnicity (race), country of
  birth (pob), region of residence (stater)
\item
  Socioeconomic: income-to-poverty ratio (povpct), education (educ),
  employment status (esr), occupation (majocc), citizenship (citizen)
\item
  Household/Social: marital status (ms), number of people living in
  household (hhnum), veteran status (vt)
\item
  Health-related: health insurance coverage status (histatus), smoking
  status (smokstat), smoked more than 100 cigarettes in lifetime
  (smok100), smokeless tobacco use (everuse)
\end{itemize}

2.1 Imbalanced outcome

Deaths account for only \textasciitilde4\% of observations: Survived (0)
378501; Died (1) 16124. Therefore, a naive classifier would achieve
\textasciitilde96\% accuracy by predicting survival for everyone. This
required explicit correction. This makes accuracy misleading and
highlights the importance of metrics such as recall, balanced accuracy,
ROC-AUC, and PR-AUC.

2.2 Categorical encoding

Categorical predictors (e.g., marital status, employment, region,
smoking status) were converted into machine-readable form using one-hot
encoding. Binary variables remained as 0/1, and continuous variables
(e.g., age, povpct) were not altered.

2.3 Handling class imbalance

LightGBM's scale\_pos\_weight parameter was set to: scale\_pos\_weight =
(\# survivors / \# deaths). This increases the penalty for
misclassifying deaths and helps the model focus on the minority class
without oversampling or synthetic data generation. A stratified 80/20
split ensured proportional representation of deaths in both sets.

\begin{enumerate}
\def\labelenumi{\arabic{enumi}.}
\setcounter{enumi}{2}
\tightlist
\item
  Methods
\end{enumerate}

A LightGBM classifier {[}1{]} was trained on the one-hot encoded feature
matrix. Model development included:

\begin{itemize}
\tightlist
\item
  Objective: binary logistic loss
\item
  Evaluation metrics: accuracy, balanced accuracy, precision, recall,
  F1, ROC-AUC, PR-AUC
\item
  Cross-validation: 5-fold CV to estimate generalization performance
\item
  Hyperparameters: tuned learning rate, max depth, number of leaves,
  number of trees
\item
  Imbalance correction: scale\_pos\_weight as described above.
\end{itemize}

3.1 Rationale for LightGBM

LightGBM is well suited for this task due to:

\begin{itemize}
\tightlist
\item
  Efficiency on very large tabular datasets
\item
  Ability to model nonlinear and interaction effects
\item
  Native handling of class imbalance
\item
  Interpretability via SHAP
\end{itemize}

\begin{enumerate}
\def\labelenumi{\arabic{enumi}.}
\setcounter{enumi}{3}
\tightlist
\item
  Results \& Discussions
\end{enumerate}

4.1 Cross-validated performance

\begin{longtable}[]{@{}lrr@{}}
\caption{Five-fold cross-validation results for
LightGBM.}\tabularnewline
\toprule\noalign{}
Metric & Mean & SD \\
\midrule\noalign{}
\endfirsthead
\toprule\noalign{}
Metric & Mean & SD \\
\midrule\noalign{}
\endhead
\bottomrule\noalign{}
\endlastfoot
Accuracy & 0.8309 & 0.0012 \\
Balanced Accuracy & 0.8361 & 0.0012 \\
Precision & 0.1745 & 0.0012 \\
Recall & 0.8418 & 0.0015 \\
F1-score & 0.2891 & 0.0017 \\
ROC-AUC & 0.9087 & 0.0022 \\
PR-AUC & 0.3497 & 0.0099 \\
\end{longtable}

4.2 Test-set performance

\begin{longtable}[]{@{}lr@{}}
\caption{Held-out test-set performance for LightGBM.}\tabularnewline
\toprule\noalign{}
Metric & Value \\
\midrule\noalign{}
\endfirsthead
\toprule\noalign{}
Metric & Value \\
\midrule\noalign{}
\endhead
\bottomrule\noalign{}
\endlastfoot
Accuracy & 0.8265 \\
Balanced Accuracy & 0.8341 \\
Precision & 0.1708 \\
Recall & 0.8425 \\
F1-score & 0.2841 \\
ROC-AUC & 0.9070 \\
PR-AUC & 0.3563 \\
\end{longtable}

The model achieved a recall of 0.8425 on the held-out test set, meaning
that it correctly identified approximately 84\% of all individuals who
actually died. In public health and clinical risk prediction settings,
this is a critical property: ``False negatives (missed deaths) are more
harmful than false positives.''

High recall indicates that the model is effective at capturing true
high-risk cases, even when they form a very small portion of the
population. This sensitivity is especially valuable for early
intervention, population screening, and prioritizing individuals for
health outreach or monitoring.

Precision was relatively low (0.1708), which is expected in datasets
where the positive class is extremely rare. Low precision does not
indicate model failure in this context; instead, it reflects the
inherent difficulty of predicting a rare outcome.

4.3 ROC Curve

\begin{figure}
\includegraphics[width=0.5\linewidth]{figures/roc_lightgbm} \caption{ROC Curve for LightGBM}\label{fig:ROC_curve}
\end{figure}

The model's ROC-AUC of 0.907 indicates excellent ability to distinguish
between individuals who died and those who survived across all possible
classification thresholds. Additionally, the PR-AUC of 0.356 is strong
given the 4\% event rate; random guessing would achieve a PR-AUC near
0.04. This demonstrates that the model is identifying high-risk
individuals far better than chance, despite class imbalance.

4.4 Key Feature Effects

To better understand how the model assigns risk, SHAP dependence
analysis was used to examine the effect of the most influential
predictors. Age showed a strong, monotonic relationship with mortality
risk: SHAP values increased steadily with age, indicating that older
individuals contribute substantially more to predicted mortality. This
effect was consistent across all income levels. Other influential
features included citizenship category, age, sex, smoking, and
income-to-poverty ratio, reflecting the combined biological and
socioeconomic determinants of mortality.

\begin{figure}
\includegraphics[width=0.5\linewidth]{figures/shap_age} \caption{SHAP dependence plot for age}\label{fig:shap-age}
\end{figure}

\begin{enumerate}
\def\labelenumi{\arabic{enumi}.}
\setcounter{enumi}{4}
\tightlist
\item
  Conclusion
\end{enumerate}

This project developed a LightGBM classifier to predict 5-year mortality
among smokers in the NLMS Tobacco Use Cohort, a large and highly
imbalanced dataset. After one-hot encoding categorical variables and
applying class weighting to address the rarity of death events, the
model demonstrated strong performance on both cross-validation and
held-out test data. High recall and AUC values indicate that the model
effectively identifies individuals at elevated mortality risk, even when
the positive class represents only \textasciitilde4\% of the population.
Although precision is low, as expected in rare-event prediction, the
model's ability to capture the majority of true deaths makes it suitable
for screening and population risk stratification. Overall, LightGBM
provides a robust, efficient, and interpretable framework for mortality
prediction in large epidemiologic datasets, offering practical utility
for public health research and targeted intervention strategies.

Reference:

{[}1{]} Ke, G., Meng, Q., Finley, T., Wang, T., Chen, W., Ma, W.,
\ldots{} \& Liu, T. Y. (2017). Lightgbm: A highly efficient gradient
boosting decision tree. Advances in neural information processing
systems, 30.

\end{document}
